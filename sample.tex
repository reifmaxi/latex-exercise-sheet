\documentclass[a4paper]{layout}

% My personal LaTeX package for exercise sheets.
\usepackage[solution]{exercise}

% Lecture specific information.
\setlogo{logo}
\setlecture{Sample~Lecture}
\setterm{Winter Term 2022/23}
\setprof{Prof.~Dr.~Empty~Set}
\setstaff{Employee~of~the~Month}
\setstaff{Some~Assistant}
\setchair{Chair~of~Fancy~Shit}

% Sheet specific information.
\setsheetnumber{1}
\settopics{New macros and environments}
\setsubmissiontime{Wed, 14 Dec, 12:00}
\setsubmissionplace{in the lecture}

% ------------------------------------ END -------------------------------------

\begin{document}

\maketitle

This document is created with the package \texttt{exercise.sty}.
It provides the following macros to create a custom document header and title
using \texttt{\textbackslash maketitle}:
\begin{itemize}
  \item
    \texttt{\textbackslash setlogo\{\}}
    to choose the logo used.
  \item
    \texttt{\textbackslash setlecture\{\}}
    and \texttt{\textbackslash setterm\{\}}
    to specify the associated lecture.
  \item[$\times$]
    \texttt{\textbackslash setprof\{\}},
    \texttt{\textbackslash setstaff\{\}},
    and \texttt{\textbackslash setchair\{\}}
    to specify involved persons.
  \item
    \texttt{\textbackslash setsheetnumber\{\}},
    \texttt{\textbackslash setcontents\{\}}
    to give information about the sheet.
  \item
  \texttt{\textbackslash setsubmissiontime\{\}},
  \texttt{\textbackslash setsubmissiondate\{\}}
  to specify the submission.
\end{itemize}
All macros are optional.
Those listed under $\times$ are intended for possible repeated use:
each usage appends the corresponding content separated by commas.
It is advisable to \texttt{\textbackslash input}
unvarying information from a separate \texttt{.tex}-file
to save time and ensure consistency across the sheets.
The sheet number can be retrieved automatically
from the file name by \texttt{retrieve-sheetnumber.tex},
if Lua\LaTeX\ and a naming convention like
\texttt{sheet\_01.tex} or \texttt{12sheet.tex}
(where the only digits correspond to the sheet number) is used.
Leading zeros thereby are ignored.

Furthermore, new environments are provided for problems and solutions,
which are shown and explained below.
The language used (English by default) can be changed to German by
\texttt{\textbackslash setcourselanguage\{german\}}.
Other languages can be added by adopting lines \texttt{34} to \texttt{43}
of \texttt{exercise.sty}.

\begin{problem}[4][The \texttt{problem} environment]
  The \texttt{problem} environment is invoked by
  \[
    \texttt{\textbackslash begin\{problem\}[<points>][<description>]}
    \quad \unicodeellipsis \quad
    \texttt{\textbackslash end\{problem\}}.
  \]
  Both \texttt{<points>} and \texttt{<description>} are optional.
\end{problem}

\begin{solution}
  The \texttt{solution} environment is invoked by
  \[
    \texttt{\textbackslash begin\{solution\}}
    \quad \unicodeellipsis \quad
    \texttt{\textbackslash end\{solution\}}.
  \]
  The contents of \texttt{solution} environments are only visible,
  if \texttt{exercise.sty} is loaded with the option \texttt{solution}.
\end{solution}

\begin{bonusproblem}[3]
  Finally there is also an environment for bonus problems,
  which behaves exactly like \texttt{problem} except for the numbering.
\end{bonusproblem}

\end{document}
